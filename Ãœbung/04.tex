\documentclass{scrartcl}

\usepackage{eurosym}
\usepackage[utf8]{inputenc}
\usepackage[ngerman]{babel}
\usepackage{verbatim}

\title{Effiziente Graphenalgorithmen - 4. Übung}
\author{Matthias Bady (210235131)\\ Christian Hildebrandt (210238835)}

\begin{document}

\maketitle

\section{Aufgabe 4.1}
Ja, bidirektionale und zielgerichtete Suche lassen sich sinnvoll kombinieren. Der Algorithmus erfolgt analog dem bidirektionalen Dijkstra.

\section{Aufgabe 4.2}
Da das asymptotische Verhalten untersucht werden soll, wird angenommen, jeder Knoten in D verfügt über jeweils K Nachfolger. Da es sich bei D um einen bipartiten Graph handelt, befinden sich diese K Nachfolger jeweils in der konträren Menge. Weiterhin soll angenommen werden, das jeder Pfad in D aus T Kanten besteht. Ausgehend vom Startknoten s enthält eine Menge damit $\sum_{i=1}^{T}{K^i}$ Knoten und die andere Menge $\sum_{i=1}^{T-1}{K^i}$ Knoten. Der Startknoten wird wegen der asymptotischen Betrachtung vernachlässigt. Damit ist die Gesamtzahl der Knoten in D und die Zahl der äußeren Iterationen $\sum_{i=1}^{T}{K^i}$. Allerdings ist die Kantenliste im letzten Durchlauf leer, so daß die Komplexität hier bei $O(K^T*0)$ liegt. Insgesamt liegt diese also bei $O((1+\sum_{i=1}^{T}{K^i} - K^T)*m) = O((\sum_{i=1}^{T-1}{K^i})*m)$. Da $\sum_{i=1}^{T-1}{K^i} \leq \sum_{i=1}^{T}{K^i}$ gilt $O(n_1*m)$

\end{document}
